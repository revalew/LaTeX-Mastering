\documentclass{article}
\usepackage[utf8]{inputenc}
\usepackage{hyperref}

\title{Pineapples\footnote{Disclaimer: The text may contain errors, so do not depend on the text for your pineapple needs.}}
\author{Taken from \url{https://en.wikipedia.org/wiki/Pineapple}}
\date{Date of copying: October 9, 2020}

\begin{document}

\maketitle

The pineapple (Ananas comosus) is a tropical plant with an edible fruit and the most economically significant plant in the family Bromeliaceae. The pineapple is indigenous to South America, where it has been cultivated for many centuries. The introduction of the pineapple to Europe in the 17th century made it a significant cultural icon of luxury. Since the 1820s, pineapple has been commercially grown in greenhouses and many tropical plantations. Further, it is the third most important tropical fruit in world production. In the 20th century, Hawaii was a dominant producer of pineapples, especially for the US; however, by 2016, Costa Rica, Brazil, and the Philippines accounted for nearly one-third of the world's production of pineapples.

Pineapples grow as a small shrub; the individual flowers of the unpollinated plant fuse to form a multiple fruit. The plant is normally propagated from the offset produced at the top of the fruit, or from a side shoot, and typically mature within a year.

\section{Botany}
The pineapple is a herbaceous perennial, which grows to 1.0 to 1.5 m (3.3 to 4.9 ft) tall, although sometimes it can be taller. In appearance, the plant has a short, stocky stem with tough, waxy leaves. When creating its fruit, it usually produces up to 200 flowers, although some large-fruited cultivars can exceed this. Once it flowers, the individual fruits of the flowers join together to create a multiple fruit. After the first fruit is produced, side shoots (called 'suckers' by commercial growers) are produced in the leaf axils of the main stem. These may be removed for propagation, or left to produce additional fruits on the original plant. Commercially, suckers that appear around the base are cultivated. It has 30 or more long, narrow, fleshy, trough-shaped leaves with sharp spines along the margins that are 30 to 100 cm (1.0 to 3.3 ft) long, surrounding a thick stem. In the first year of growth, the axis lengthens and thickens, bearing numerous leaves in close spirals. After 12 to 20 months, the stem grows into a spike-like inflorescence up to 15 cm (6 in) long with over 100 spirally arranged, trimerous flowers, each subtended by a bract.

The ovaries develop into berries, which coalesce into a large, compact, multiple fruit. The fruit of a pineapple is usually arranged in two interlocking helices, Typically there are eight in one direction and 13 in the other, each being a Fibonacci number.

The pineapple carries out CAM photosynthesis, fixing carbon dioxide at night and storing it as the acid malate, then releasing it during the day aiding photosynthesis.

The pineapple comprises five botanical varieties, formerly regarded as separate species:
\begin{itemize}
    \item  Ananas comosus var. ananassoides
    \item Ananas comosus var. bracteatus
    \item Ananas comosus var. comosus
    \item Ananas comosus var. erectifolius
    \item Ananas comosus var. parguazensis
\end{itemize}
\subsection{Pollination}
In the wild, pineapples are pollinated primarily by hummingbirds. Certain wild pineapples are foraged and pollinated at night by bats. Under cultivation, because seed development diminishes fruit quality, pollination is performed by hand, and seeds are retained only for breeding. In Hawaii, where pineapples were cultivated and canned industrially throughout the 20th century, importation of hummingbirds was prohibited.
\section{English name}
The first reference in English to the pineapple fruit was the 1568 translation from the French of André Thevet's The New Found World, or Antarctike where he refers to a Hoyriri, a fruit cultivated and eaten by the Tupinambá people, living near modern Rio de Janeiro, and now believed to be a pineapple. Later in the same English translation, he describes the same fruit as a Nana made in the manner of a Pine apple, where he used another Tupi word nanas, meaning "excellent fruit". This usage was adopted by many European languages and led to the plant's scientific binomial Ananas comosus, where comosus, "tufted", refers to the stem of the plant. Purchas, writing in English in 1613, referred to the fruit as Ananas, but the OED's first record of the word "pineapple" itself by an English writer by Mandeville in 1714.
\end{document}
