\documentclass{article}
\usepackage[utf8]{inputenc}
\usepackage{amsmath, amssymb, amsthm}

\title{Basic Mathematics}
\author{Eirik and Stine }
\date{September 2020}

\begin{document}

\maketitle

\section{Introduction}

Inline math $f(x) = 5x / 3 = \frac{5x}{3}$. Everyone knows than $2 + 2 \neq 5$. The square root $\sqrt[3]{9}$ is $3$. A Greek symbol is $\Pi$. Another example is $\Lambda$. Display math is 
\[f(x) = 5x / 3 = \frac{5x}{3}\] 
More writing.... A second degree polynomial is on the form 
\[f(x) = a_{2} x^{2} + a_{1}x + a_{00}.\]
$\sin(2)$. We also have $\cos(\pi) = -1$. An important trigonometric identity is 
\[\cos(x)^{2} + \sin(x)^{2} = 1.\]
We have that $A \subseteq B$.

\section{Exercise - Zeros of Second Degree Polynomial}

The (real) zeros of the second degree polynomial $f(x) = ax^2 + bx + c$ is either:
\begin{itemize}
    \item On the form 
    \[\frac{-b \pm \sqrt{b^{2} - 4ac}}{2a},\]
        when there are two (real) zeroes.
    \item On the form 
        \[\frac{-b}{2a},\]
        when there is one (real) zero.
    \item There are no real zeroes.
\end{itemize}

\end{document}
