\documentclass[12pt,titlepage]{article}

\usepackage{amssymb, amsmath, amsthm, amsfonts} % standard package combo
\newtheorem{theorem}{Theorem}[section] % creating the theorem env
\newtheorem{definition}[theorem]{Definition}
\newtheorem*{remark}{Remark}

\title{Beautiful Mathematics Part 2}
\author{Me}
\date{\today}

\begin{document}
\maketitle
\newpage
\tableofcontents
\newpage

\section{More Math Symbols}
\subsection{Sum}
$\sum_{n=0}^{\infty}\frac{1}{n^{2}+1}$
\[\sum_{n=0}^{\infty}\frac{1}{n^{2}+1}\]

\subsection{Integrals}
$\int_{0}^{\infty}x^{2}dx$\\
$\iint_{D}^{}x^{2}+ydydx$
\[\int_{0}^{\infty}x^{2}dx\]

\subsection{Limits}
$\lim_{x \to 1}x$
\[\lim_{x \to 1}x\]

\subsection{Other}
$\max_{n \in \{1, 2, 3\}}n \\ \min, \sup, \inf, \limsup$
\[\max_{n \in \{1, 2, 3\}}n\]

\section{Exercise: Replicate the Mathematics 1}
\[\sum_{n=1}^{\infty}\frac{1}{2^{n}}=1\]
\[\iint_{D}e^{-x^{2}-y^{2}}\,dx\,dy\]

\newpage
\section{Mathematics on Multiple Lines}
\subsection{Multiline Equations}
\begin{multline} \label{multiline_env}
    % normaly the equation goes out of bounds so we use \\
    % to remove the numbering of the equation we use the *
    f(x) = 2x+2x^2+2x+2x^2+2x+2x^2+2x+2x^2+2x+\\2x^2+2x+2x^2+2x+2x^2+2x+2x^2+\\2x+2x^2+2x+2x^2+2x\\+2x^2+2x+2x^2+2x+2x^2+2x+2x^2
\end{multline}

\subsection{Align env}
\begin{align} \label{align_env}
    % the &= means we want to align the equation with the = sign. Alignment on the eq sign.
    2x+3y+3z+w&=2 & 3x+4y+1z+0w&=4 \\
    x+4y+3z+w&=3 & -x-y-z-w&=3 \notag % removes the numbering on 1 line
\end{align}

In the \eqref{multiline_env} we can see how the equation is spanning Multiple lines.

In the \eqref{align_env} we can see how we can align 2 equations on a single line

\subsection{Simplifying the equation with Align env}
\begin{align*}
    f(x)&=2x+3y-2x\\
    &= 3y
\end{align*}

\section{Formatting Mathematics}
Rule of thumb is that the inline Math in "\$ \$" env, should not exceed the $\frac{1}{3}$ of the text width. If it is longer than that we should be using a display Math in "\textbackslash[ \textbackslash]" env.
\vspace{0.5cm}

We can also change the size of the parenthesees with "\textbackslash big, \textbackslash Big, \textbackslash bigg, \textbackslash Bigg" like this $\longrightarrow \; \Bigg( \, \bigg( \, \Big( \, \big($.

Also it is a good practice to leave a little space between the orgument of the "$\int$" and the "$dx$". So instead of $\int xdx$ we do "$\int x\,dx$" with "\textbackslash ," space (small space).

\section{Exercise: Replicate the Mathematics 2}
\begin{align*}
    \int_{0}^{1} \frac{(x+1)(x-2)}{x-3}\,dx &= \int_{0}^{1}\frac{x^2-x-2}{x-3}\,dx\\
    &= \int_{0}^{1} x+2+\frac{4}{x-3}\,dx\\
    &= \bigg[\frac{1}{2}x^2+2x+4\ln \vert x-3\vert\bigg]_{0}^{1}\\
    &= \frac{5}{2}+4 \ln (2/3).
\end{align*}

\section{Math Fonts}
\subsection{Boldface Math Font}
\[a \rightarrow \mathbf{a}\]
\subsection{Caligraphy fonts}
\[T \rightarrow \mathcal{T}\]

\subsection{Blackboard Boldface}
\[R \rightarrow \mathbb{R}\]

\subsection{Symbols}
\[\bar{a}, \overline{ab}\]
\[\dot f, f', f''', f^{(3)}\]

\newpage
\section{Matrices and Cases}
These envs need to be written inside of the math env (inline / display math, align, multiline or any other which accepts the math). It is not advised to write Matrices in inline math env.
\subsection{Matrices}
\[
    A = 
    \begin{bmatrix}
        % we can start this env with letters before the "matrix"
        % these letters determine the parentheses type
        % p = ()     b = []     v = ||
        1 & 2 & 3 & 4 \\
        5 & 6 & 7 & 8 
    \end{bmatrix}
\]

\subsection{Cases}
\[
    |x| = 
    \begin{cases}
        x & \text{ for } x > 0 \\
        0 & \text{ for } x = 0 \\
        -x & \text{ for } x < 0
    \end{cases}
\]

\section{Exercise: Replicate the Mathematics 3}
\[
    \mathbf{v}=
    \begin{pmatrix}
        1 \\ 2 \\ 3
    \end{pmatrix}
    \qquad A = 
    \begin{bmatrix}
        1 & 0 & 1 \\
        0 & 1 & 1 \\
        0 & 0 & 1
    \end{bmatrix}
\]  

\section{Exercise: Replicate the Mathematics 4}
\[
    f(x)=
    \begin{cases}
        1, & \text{ for } x > 0 \\
        0, & \text{ for } x = 0 \\
        -1, & \text{ for } x < 0
    \end{cases}
\]

\newpage
\section{Making Theorems, Definitions, and Remarks} \label{section: Making Theorems}
We have to create the Theorem env (in the preamble) with the "amsthm" package.

\begin{theorem}[Name of theorem] % in the options bracket we can write the name / reference of this specific theorem
    This is a Theorem.
\end{theorem}

\begin{definition} \label{def: This is definition}
    This is a definition.
\end{definition}

As we can see the theorem envs we created are all numbered. To change that we use "*" in the env definition (preamble).

\begin{remark}
    This is a remark.
\end{remark}

We can also label the theorem envs. \\
The definition: \ref{def: This is definition}\\

\section{Numbering and Theoremstyle}
We can see that the definitions and theorems have their own individual counting (Section \ref{section: Making Theorems}). We can change that with the option in preamble like so: "\textbackslash newtheorem\{definition\}[theorem]\{Definition\}" where the "theorem" is the name of the env of which we want to continue the counting.

We can also include the number of the section when writing the theorem envs with the option "section" like this: "\textbackslash newtheorem\{theorem\}\{Theorem\}[section]". We can also use options like "chapter" when we are writing a book, or "subsection" to include the number of the subsection as well as section.

We can also change the theorem style with the "\textbackslash theoremstyle\{\}" and options: "plain" which is the default style, "definition" - the text is no longer italicised, "remark" - the text is italicised instead of being Boldface. These styles are applicable to all the "\textbackslash newtheorem\{\}" underneath the new style.

\vspace{0.5cm}
With the theorem we also need \textit{proof}.
\begin{proof}
    This is a proof.
\end{proof}

\end{document}