\documentclass[12pt, titlepage]{article} 

\usepackage{calligra} % font
\usepackage[T1]{fontenc} % encoding for fonts
\usepackage[dvipsnames]{xcolor} % get more colors with this option
\definecolor{myNewColor}{RGB}{187, 42, 97}
\definecolor{nextColor}{HTML}{ddaadd} % code in hex

\usepackage{hyperref}

\title{{\ttfamily \Huge Styling with {\calligra Fonts} and {\color{nextColor}Colors}}}
\author{Me}
\date{\today}

\begin{document}
\maketitle

\newpage
\thispagestyle{empty}
\tableofcontents
\newpage

\section{Manipulating Text}
\textbf{Hello}\textit{there}, \underline{general} Kenobi.
\subsection{Sans Serif font}
{\sffamily Sans Serif text}

\subsection{Monospace / Typewritter font}
{\ttfamily Monospace font}

\subsection{Size of the text}
\begin{itemize}
    \item tiny: {\tiny Hello}
    \item scriptsize: {\scriptsize Hello}
    \item footnotesize: {\footnotesize Hello}
    \item small: {\small Hello}
    \item normalsize: {\normalsize Hello}
    \item large: {\large Hello}
    \item Large: {\Large Hello}
    \item LARGE: {\LARGE Hello}
    \item huge: {\huge Hello}
    \item Huge: {\Huge Hello}
\end{itemize}

\section{{\calligra Finding Other Fonts}}
We can find defferent fonts \href{https://tug.org/FontCatalogue/}{\textbf{here}}.

\newpage
\section{\color{red}{Setting Basic Colors}}
We do that by using the "\textbackslash usepackage[dvipsnames]\{xcolor\}" {\color{OliveGreen}command}.

\section{Design Custom Colors}
Resources: 
\begin{itemize}
    \item \href{https://hackernoon.com/hex-colors-how-do-they-work-d8cb935ac0f}{HEX colors - how do they work}.
    \item \href{https://coolors.co/655a7c-ab92bf-afc1d6-cef9f2-d6ca98}{Color picker}
\end{itemize}
We can define our Custom color in the preamble with \\
"\textbackslash definecolor\{myNewColor\}\{RGB\}\{187, 42, 97\}". RGB has to be capitalized.\\
Now we can use it wit "\textbackslash textcolor\{myNewColor\}\{text\}" like \textcolor{myNewColor}{this}.\\
We can also define colors using HEX with "\textbackslash definecolor\{nextColor\}\{HTML\}\{ddaadd\}". The HTML means we are using HEX {\tiny (for some reason...)}.
After defining it we can use it with the same command like \textcolor{nextColor}{this}.


\end{document}