\documentclass[12pt, titlepage]{article}

\usepackage{hyperref}
\usepackage{amsmath, amssymb, amsthm}

\title{Beautiful Mathematics Part 1}
\author{Me}
\date{\today}

\begin{document}
\maketitle

\newpage
\tableofcontents
\newpage

\section{Inline Mathematics}
Inline math: $f(x) = 5x + 3$. \\
Everyone knows that $2 + 2 \neq 5$. \\
The square root: $\sqrt[3]{2}$.

\section{Greek letters and fractions}
To write the Greek letters we use Inline Math: $\alpha \pi \Pi$ \\[0.5em]
Same with fractions: $\frac{1}{2}$

\section{Display Math and Subscripts}
Display Math is a math displayed on a separate line \& in center.
\[f(x) = 5x + 3\]
Superscript: $ax^2$ \\
Subscript: $a_1x$

\section{Trigonometric Functions}
We write the trigonometric functions using inline math again. \\
sinus: $\sin{2}$ \\
cosinus: $\cos{\pi}$
\[\sin^{2}{x} + \cos^{2}{x} = 1\]

\newpage
\section{Exercise: Second Degree Polynomial}
The real zeros of the second Degree Polynomial $f(x) = ax^2 + bx + c$ is either:
\begin{itemize}
    \item on the form
    \[\frac{-b \pm \sqrt{b^2 - 4ac}}{2a}\]
    when there are two real zeros.
    \item on the form
    \[\frac{-b}{2a}\]
    when there is one real zero.
    \item There are no real zeros
\end{itemize}

\end{document}
