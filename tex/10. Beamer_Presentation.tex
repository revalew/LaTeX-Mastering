\documentclass[10pt,titlepage]{article}

\usepackage{hyperref}

\title{Using Classes and Templates}
\author{Me}
\date{\today}

\begin{document}
\maketitle

\newpage
\thispagestyle{empty}
\tableofcontents
\newpage

\section{Beamer Basics}

\begin{verbatim}
    \documentclass{beamer}

    \title{Using Classes and Templates}
    \author{Me}
    \date{\today}
    \institute{A University}
    
    \begin{document}
    \maketitle

    maketitle works the exact same way, but we can also use the 
    "\frame{\titlepage}".
    
    We also use the "\institute{A University}" t ospecify the institute (???).

    \begin{frame}{Content}
        \tableofcontents       
        We create the toc based on the sections even though those 
        are not visible in the frane.
    \end{frame}

    \section{Introduction}
    \begin{frame}{Titile}
        % \frametitle{<title>}
        Some text.
    \end{frame}
    
    \end{document}
    
\end{verbatim}

\newpage
\section{How to Structure your Beamer}

We are using the block environments inside our beamer.

\begin{verbatim}
    \begin{block}{Theme}\pause % title of the block
        Here are the different things.
    \end{block}

    \begin{alertblock}{Important}
        This block is used for important information.
    \end{alertblock}

    \begin{example} \pause
        This is an example block, which does not need a title.
        This block can take an optional argument like "[First Example]",
        which will display this text in the color matched "()".
    \end{example}

    We are using "\pause" as a next slide indicator. We can display one block
    on the page and then display like 2 or 3 on the next even though all of them
    are on the same page. It splits the frame into two frames,
    where the second starts right after the \pause command

    We can also use "\alert{text}" to highlight a text in the beamer
    (red but I think this is customizable).
\end{verbatim}

\section{Styling Your Beamer}

\subsection{Theme}
\begin{verbatim}
    We can change the theme by specifying the thene name in the preamble.
    "\usetheme{}" like: \usetheme{Hannover} or Bergen.

    We can also change only the color with "\usecolortheme{}" like:
    beetle or fly.
\end{verbatim}
We can also search the gallery which can be accessed here: \\
\url{https://deic.uab.cat/~iblanes/beamer_gallery/index.html} \\
or here: \\
\url{https://no.overleaf.com/latex/templates/tagged/presentation}

\section{Project: Make a Beamer Presentation}

This project will be done in a separate file named "10. Beamer\textunderscore Project.tex".

\end{document}